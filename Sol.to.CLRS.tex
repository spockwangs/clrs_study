%%%%%%%%%%%%%%%%%%%%%%%%%%%%%%%%%%%%%%%%%%%%%%%%%%%%%%%%%%%%%%%%%%%%%
% Solutions to "Introduction to Algorithms", Second Edition.
% by wbb
%%%%%%%%%%%%%%%%%%%%%%%%%%%%%%%%%%%%%%%%%%%%%%%%%%%%%%%%%%%%%%%%%%%%%

\documentclass[11pt]{article}
\usepackage{fullpage}
\usepackage{clrscode}
\usepackage{amsmath, amsfonts}

%%%%%%%%%%%%%%%%%%%%%%%%%%%%%%%%%%%%%%%%
% This is the preamble.
%%%%%%%%%%%%%%%%%%%%%%%%%%%%%%%%%%%%%%%%
\author{wbb}
\title{Solutions to CLRS}
\renewcommand{\today}{March 24, 2010}% The date when this file written.

\newcommand{\Exercise}[1]
    {\begin{flushleft}{\large \textbf{Exercise #1}}\end{flushleft}}
      

%%%%%%%%%%%%%%%%%%%%%%%%%%%%%%%%%%%%%%%
% The text begins here.
%%%%%%%%%%%%%%%%%%%%%%%%%%%%%%%%%%%%%%%
\begin{document}
\maketitle

\section*{Chapter 24}

\Exercise{24.1-5}

The desired value for each $v$ in $V$ is similar to the $\delta(v)$ in the
original Bellman-Ford algorithm.  It has the similar property such as
optimal substructure, upperbound property, convergence property and
path-relaxation property.  So it can be solved using Bellman-Ford algorithm
with the $d$ function changed.
  
\Exercise{24.4-5}

In fact, the added dummy source $s$ is not really needed.  We just need to
modify the Bellman-Ford algorithm to initialize all $d[v]$ to 0 and it will
work well.

Why? The Bellman-Ford algorithm has 2 proedures: relax edges and
check negative-weight cycles.  We will prove that the algorithm has
the same effects without the added edges $(s, v)$ as with the added
edges in the 2 procedures.  When doing relaxation, each added edge
$(s, v)$ is used only once to relax each $d[v] (v \in V)$ to 0 then it
will not be used to relax any edge because after having relaxed $d[v]$
to 0 the relax condition $d[v] < d[s] + w(s, v)$ (simplified to $d[v]
< 0$) will not be satisfied again.  So we initialize all $d[v]$ to 0
and the edges $(s, v)$ will be useless.  When checking negative-weight
cycles, the conditon $d[v] > d[s] + w(s, v)$ is not satisfied whether
or not negative weight cycles exist.  So the added edges $(s, v)$
is not needed to check the existence of negative-weight cycles.

%%%%%%%%%%%%%%%%%%%%%%%%%%%%%%%%%%%%%%%%%%%%%%%%%%%%%%%%%%%%%%%%%%%%%%%%%%
% The Bibliography
%
% \begin{thebibliography}{widest entry}
%   \bibitem[label1]{cite_key1} bibliographic info
%   \bibitem[label2]{cite_key2} bibliographic info
%    ...
% \end{thebibliography}
% 
% We can reference the first bibliography item using \cite{cite_key1} 
% in the text.
%%%%%%%%%%%%%%%%%%%%%%%%%%%%%%%%%%%%%%%%%%%%%%%%%%%%%%%%%%%%%%%%%%%%%%%%%%

\end{document}

